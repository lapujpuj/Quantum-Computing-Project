\documentclass[12pt]{article}
\usepackage[utf8]{inputenc}
\usepackage{graphicx}
\usepackage{amsmath}
\usepackage{hyperref}
\usepackage{float}
\usepackage{geometry}
\usepackage{titling}
\usepackage{abstract}
\usepackage{multicol}
\usepackage{biblatex}
\geometry{a4paper, left=2cm, right=2cm, top=2cm, bottom=2cm}

% \addbibresource{bibliography.bib}

\begin{document}

\title{Projet de M1 - Informatique Quantique - Jalon 2}
\author{Martin PUJOL \\ martin.pujol@etu.sorbonne-universite.fr}
\date{}
\maketitle
Le jalon 2 présente une mise à jour de mon projet bibliographique sur l'informatique quantique. Après une réflexion personnelle et des échanges avec les enseignants, j'ai décidé de me concentrer sur l'optimisation quantique. J'ai également ajouté une section comparant les technologies quantiques actuelles, et j'ai résolu le problème du Max Cut Edge en utilisant un ordinateur quantique. J'ai consigné ces avancées dans mon rapport, qui me semble proche de sa forme finale. Le rapport est disponible sur GitHub à l'adresse suivante : \url{https://github.com/lapujpuj/Quantum-Computing-Project}.
Le plan final du projet est organisé comme suit :

\section{Qu’est-ce qu’un qubit ?}
\begin{itemize}
    \item Définition
    \item Mesure d’un qubit
    \item Superposition et intrication de qubits
\end{itemize}

\section{Comment produire un qubit et agir sur son état ?}
\begin{itemize}
    \item Les qubits à ions piégés
    \item Les qubits supraconducteurs
    \item Les qubits à photons
    \item Quelle technologie choisir ?
\end{itemize}


\section{Algorithme et optimisation quantique}
\subsection{Les portes quantiques}
\begin{itemize}
    \item Porte de Hadamard
    \item Porte de Pauli
    \item Porte de Toffoli
\end{itemize}
\subsection{Optimisation quantique}
\begin{itemize}
    \item Théorème adiabatique
    \item Hamiltonien d’Ising
\end{itemize}

\section{Résoudre un problème d’optimisation : Le Max Cut Edge}
\begin{itemize}
    \item Encodage quantique du problème : Quel Hamiltonien ?
    \item Implémentation et resultats
\end{itemize}


\section{Conclusion}

Ce projet bibliographique a permis de présenter les principes fondamentaux de l'informatique quantique, en particulier la notion de qubit, les différentes façons de les produire et les problèmes liés à leur utilisation. Le projet a également abordé la résolution d'un problème d'optimisation classique à l'aide d'un ordinateur quantique.

Les qubits, en tant qu'unités de base de l'information quantique, ont des propriétés uniques telles que la superposition et l'intrication, qui permettent d'effectuer des calculs de manière beaucoup plus efficace que les bits classiques. Cependant, la mesure d'un qubit est un processus irréversible qui détruit la superposition, ce qui rend la manipulation des qubits très délicate.

Pour produire des qubits, il existe plusieurs technologies prometteuses, notamment les qubits supraconducteurs, les qubits à ions piégés et les qubits à photons. Chacune de ces technologies a ses avantages et inconvénients en termes de stabilité, de scalabilité et de facilité de manipulation.

Le projet a également abordé la résolution d'un problème d'optimisation classique, le Max Cut Edge, à l'aide d'un ordinateur quantique. Ce problème consiste à trouver la coupe d'un graphe qui maximise le nombre d'arêtes coupées. En utilisant un algorithme quantique, il est possible de trouver la solution optimale de manière beaucoup plus efficace que les algorithmes classiques.

En conclusion, l'informatique quantique est un domaine en pleine expansion qui offre des perspectives prometteuses pour la résolution de problèmes complexes. Cependant, la manipulation des qubits reste un défi majeur, et il est important de poursuivre les recherches pour améliorer la stabilité et la scalabilité des technologies de production de qubits. Le projet a permis de présenter les principes fondamentaux de l'informatique quantique et de montrer comment un ordinateur quantique peut être utilisé pour résoudre un problème d'optimisation.


\end{document}