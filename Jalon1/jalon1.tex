\documentclass[12pt]{article}
\usepackage[utf8]{inputenc}
\usepackage{graphicx}

\usepackage{amsmath}
\usepackage{hyperref}
\usepackage{float}
\usepackage{geometry}
\usepackage{titling}
\usepackage{abstract}
\usepackage{multicol}
\usepackage{biblatex}

\geometry{a4paper, left=2cm, right=2cm, top=2cm, bottom=2cm}

% \addbibresource{bibliography.bib}

% \usepackage{fancyhdr}

% \pagestyle{fancy}
% \fancyhf{}
% \lhead{M1 PAD \\ \rule{\textwidth}{0.4pt} \\ Projet de M1}
% \rhead{MU4PYD? \\ \rule{\textwidth}{0.4pt} \\ 2023-2024}
% \renewcommand{\headrulewidth}{0pt}


\begin{document}

\title{Projet de M1 - Informatique Quantique - Jalon Initial}
\author{Martin PUJOL \\ martin.pujol@etu.sorbonne-universite.fr}
\date{}

\maketitle

\section*{Introduction}

Ce document présente le jalon 1 de mon projet bibliographique sur l'informatique quantique. Il comprend les éléments suivants :

\begin{itemize}
    \item \textbf{Mise en place du projet et choix du sujet}
    \item \textbf{Méthodologie et approche du sujet}
    \item \textbf{Plan du projet}
\end{itemize}

\section{Mise en place du projet et choix du sujet}

J'ai proposé un sujet n'existant pas sur la plateforme car je porte un grand intérêt aux technologies quantiques. Ce sujet très attractif dans le monde de la physique est aussi abstrait et j'ai souhaité découvrir ce sujet en y apportant une dimension plus ludique en y joignant une partie pratique. Ce projet comporte donc deux parties :

\begin{itemize}
    \item \textbf{Partie bibliographique} : elle portera sur les moyens de produire et de contrôler les qubits.
    \item \textbf{Partie pratique} : elle consistera à optimiser un problème connu (Max Cut Edge) en utilisant un ordinateur quantique.
\end{itemize}

\section{Méthodologie et approche du sujet}

Ma méthodologie s'articule autour des points suivants :

\begin{itemize}
    \item \textbf{Revue de la littérature scientifique} : je m'appuierai sur des articles scientifiques, des livres et des conférences pour acquérir une connaissance approfondie du sujet. En particulier, le cours "Note for Quantum Computing" dispensé à l'université de Chalmers en Suède.
    \item \textbf{Expérimentation et simulations} : j'utiliserai des plateformes de simulation quantique pour mettre en pratique mes connaissances et tester des algorithmes quantiques. Je m'appuie en particulier sur les notebooks proposés par IBM.
    \item \textbf{Collaboration et communication} : J'ai déjà partagé mes avancées et souhaite continuer à le faire en sollicitant régulièrement mes professeurs qui m'encadrent.
\end{itemize}

\section*{Abstract du projet}

Ce projet bibliographique souhaite succinctement présenter les principes fondateurs derrière l'ordinateur quantique, aborder la notion de qubits, la façon d'en produire et les problèmes liés à leur utilisation. Par la suite, le projet se veut un peu plus ludique en formalisant un problème d'optimisation classique (simpliste) et en le résolvant à l'aide d'un ordinateur quantique.

\section{Plan du projet}

Le projet sera divisé en plusieurs sections :

\begin{itemize}
    \item \textbf{Introduction} : elle présentera le contexte et les objectifs du projet.
    \item \textbf{Partie bibliographique} :
        \begin{itemize}
            \item Définition et propriétés des qubits
            \item Production et manipulation des qubits
            \item Comparaison / Benchmark des technologies utilisées
            \item Algorithmes quantiques fondamentaux
        \end{itemize}
    \item \textbf{Partie pratique} :
        \begin{itemize}
            \item Définition du problème Max Cut Edge
            \item Formulation du problème en tant que problème d'optimisation quantique
            \item Implémentation et simulation de l'algorithme quantique
            \item Analyse des résultats et comparaison avec les approches classiques
        \end{itemize}
    \item \textbf{Conclusion} : elle synthétisera les résultats obtenus et proposera des perspectives de recherche future.
\end{itemize}

\section{Conclusion}

Ce jalon 1 présente les bases de mon projet bibliographique sur l'informatique quantique. Je suis convaincu que ce sujet est porteur d'un grand potentiel et j'ai hâte de poursuivre mes recherches dans ce domaine.

\section*{Bibliographie}


\begin{itemize}
    \item \textbf{Kockum, A. F., Soro, A., García-Álvarez, L., Vikstål, P., Douce, T., Johansson, G., \& Ferrini, G. (2023). Lecture notes on quantum computing.} 
    \item \textbf{Preskill, J. (n.d.). Physics 219 Course Information.} 
    \item \textbf{Steane, A. M. (1998). Quantum Computing.}
    \item \textbf{Wintersperger, K., Dommert, F., Ehmer, T., Hoursanov, A., Klepsch, J., Mauerer, W., Reuber, G., Strohm, T., Yin, M., \& Luber, S. (2023). Neutral atom quantum computing hardware: performance and end-user perspective.} \cite{wintersperger_neutral_2023}
    \item \textbf{Building Quantum Computers with Neutral Atoms.} 
    \item \textbf{Introduction to Transmon Physics.}
    \item \textbf{Stęchły, M. (2024). mstechly/quantum\_tsp\_tutorials.} 
    \item \textbf{Jain, S. (2021). Solving the Traveling Salesman Problem on the D-Wave Quantum Computer.}
    \item \textbf{Nielsen, M. A., \& Chuang, I. L. (2012). Quantum Computation And Quantum Information 10th Anniversary Edition.} 
\end{itemize}

\section*{Prochaines étapes}

\begin{itemize}
    \item Finaliser la revue de la littérature scientifique sur les qubits.
    \item Inclure la partie sur la comparaison des qubits en développement.
    \item Définir plus précisément le problème Max Cut Edge et sa formulation en tant que problème d'optimisation quantique.
    \item Commencer l'implémentation et la simulation de l'algorithme quantique.
\end{itemize}


\end{document}
